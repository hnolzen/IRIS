\documentclass[a4paper, 11pt]{scrartcl}
\usepackage[top = 2.5cm, bottom = 2cm, left = 2.5cm, right = 2.5cm]{geometry}
\usepackage[onehalfspacing]{setspace}
\usepackage{mathtools}
\usepackage{fontspec}
\usepackage[english]{babel}
\usepackage{hyperref}
\hypersetup{
	colorlinks = true,
	linkcolor  = black,
	citecolor  = blue,
	urlcolor   = blue
}

\usepackage{acronym}
\usepackage{enumerate}
\usepackage{tabularx}
\usepackage{booktabs}

\usepackage{csquotes}
\usepackage[backend = biber, style = apa]{biblatex}
\addbibresource{../iris_literature.bib}

\setlength{\parskip}{1em}

\title{IRIS - ODD Protocol}
\author{}
\date{\today}

\begin{document}

\maketitle
\tableofcontents

\newpage
\listoffigures
\listoftables

\section*{List of Acronyms}
\begin{acronym}
\acro{ODD}{Overview, Design Concepts and Details}
\end{acronym}


\section{Overview}
This model description follows the Overview, Design concepts and Details (ODD) protocol. ODD is a standard format to systematically
describe models in a comprehensive and transparent way aiming to allow for model replication ~\parencite{Grimm.2010, Grimm.2020}.

IRIS is a cohort-based and spatially-explicit population model to simulate local dynamics of \textit{Ixodes ricinus} ticks under
changing climate parameters. It simulates the spatial and temporal development and questing activity of ticks of different life stages
over a year. The population dynamic is driven by climate predictors such as temperature and humidity and by ecological predictors such
as habitat suitability. The IRIS model is to be expanded in the future to include a Lyme disease dynamic.

This model has been developed within project `Infectious diseases and allergies` of the HI-CAM initiative
~\footnote{\url{https://www.helmholtz-klima.de/en/adaptation/project-infectious-diseases-and-allergies}}.


\subsection{Purpose}
The overall purpose of the IRIS model is to understand the impact of changing climate on local tick and borrelian dynamics. As
ticks are highly relevant vectors of Lyme disease, it is important to gain a better understanding of their future climate-dependent
dynamics.


\subsection{Entities, state variables, and scales}
The model is composed of the following entities: (a) sub populations of \textit{Ixodes ricinus} ticks belonging to a cohort and (b) grid
cells forming an idealised model landscape. Each cohort represents a tick life cycle stage together with a behavioural state. Three tick
life cycle stages are possible: (1) larvae, (2) nymphs and (3) adults. Furthermore, four behavioural states are possible: (i) questing,
(ii) inactive, (iii) fed and (iv) late fed. Hence, in total there are 12 tick cohorts present in the model. Grid cells serve as habitat for
tick sub populations. Three different idealised habitat types exist in the model: (a) forest, (b) ecotone and (c) meadow.

The spatial scale of the model corresponds to a local region of approximately 14400 $m^{2}$ ($=$ 0.0144 $km^{2}$) in reality. The model
idealised landscape consists of 12 \times 12 $=$ 144 grid cells. A grid cell corresponds to an area of 10 $m$ \times 10 $m$ $=$ 100
$m^{2}$. This grid size was chosen because tick densities are usually expressed in numbers per 100 $m^{2}$
(See, e.g. \ ~\cite{Boehnke.2015, Brugger.2016, Brugger.2018}).

The IRIS model runs with daily time steps. One model run simulates one year, from January 1 to December 31, in reality. On this time
scale, we are able to capture the local population dynamics of ticks well. Since our model is cohort-based we are looking at averaged
behaviour which harmonises well with daily time steps. Required input data such as temperature and humidity is  available on a daily
basis ~\parencite(DWD.2020, GERICS.2020). Furthermore, daily time steps are also computationally feasible.

% Insert figure with graphical summary with entities and state variables here


\subsection{Process overview and scheduling}
The IRIS model is executed in daily discrete time steps. Since the model always simulates one year, 365 days are usually simulated, unless
a year is a leap year. In a leap year 366 days are simulated including 29 February. In each time step, the the following model processes
are executed and state-variables are updated:

\begin{enumerate}
	\item Development to the next life cycle stage
	\item Freezing
	\item Desiccation
	\item Feeding: Attaching of ticks to host animals and translocation
	\item Weather
	\item Activity
\end{enumerate}


\section{Design}


\subsection{Basic principles}
Spatially explicit stochastic cohort-based population model
Entity-Component-System

\subsection{Stochasticity}
To determine random numbers we use the seedable random number generator ``Mersenne-Twister'' provided by Apache Commons Math library
(version 3.6.1.). In addition we implemented a method roundRandom() to allow processes that statistically affect less than
a single tick to still happen.
% tbc. needs further explanation.
% insert source code of roundRandom() here


\section{Details}

\subsection{Globals}

\subsection{Initialisation}

\subsection{Input data}


\begin{table}[h!]
\caption{Input parameters}
\label{input_parameters}
\begin{tabularx}{\textwidth}{llll}
\toprule
\textbf{Parameter} & \textbf{Symbol} & \textbf{Type}     & \textbf{Description}       \\
\midrule
meanTemperature    & $t_{mean}$      & double            & Daily mean temperature     \\
minTemperature     & $t_{min}$       & double            & Daily max temperature      \\
maxTemperature     & $t_{max}$       & double            & Daily min temperature      \\
humidity           & $h$             & double            & Daily humidity             \\
\midrule
fructificationIndex & $f$            & int               & Fructification index       \\
\bottomrule
\end{tabularx}
\end{table}


\subsection{Submodels}

\subsubsection{Development}

\subsubsection{Desiccation}

\subsubsection{Freezing}

\subsubsection{Activity}

\subsubsection{Feeding}


\newpage
\printbibliography[heading = bibintoc, title = {Bibliography}]

\end{document}
