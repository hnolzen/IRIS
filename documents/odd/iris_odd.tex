\documentclass[a4paper, 11pt]{article}
\usepackage{geometry}
\geometry{top = 2cm, bottom = 2cm, left = 2cm, right = 2cm}
\usepackage[english]{babel}
\usepackage[utf8]{inputenc}
\usepackage[T1]{fontenc}
\usepackage{amsmath}
\usepackage[colorlinks = true, allcolors = blue]{hyperref}

\title{IRIS - ODD Protocol}
\author{}
\date{\today}

\begin{document}	
\maketitle
\tableofcontents

\section{Overview}
IRIS is a population model for local dynamics of \textit{Ixodes ricinis} ticks under changing climate parameters.
We plan to add a transmission model for borrelian dynamics in the near future.


\subsection{Purpose}
The purpose of the model is to understand the impact of changing climate parameters on local tick and borrelian dynamics. 


\subsection{Entities, state variables, and scales}

\subsubsection{Entities}
Idealised model landscape consisting of grid cells 
habitats (forest, ecotone, pasture)
Tick population consisting of cohorts of life cycle stages (larvae, nymphs, adults)


\subsubsection{State variables}

list of state variables (tba.)


\subsubsection{Scales}
1 simulation run = 1 year 
daily time steps

landscape 144 (12 x 12) grid cells
1 grid cell = 100 sqm 


\subsection{Process overview and scheduling}

\section{Design concepts}

\subsection{Basic principles}
Spatially explicit stochastic cohort-based population model
Entity-Component-System


\subsection{Stochasticity}
seedable random number generator: 
Mersenne Twister implementation 
provided by Apache Commons Math library

roundRandom


\section{Details}

\subsection{Globals}


\subsection{Initialisation}
initial number of ticks for all life cycle and activity stages 
initial seed number
initial fructification index (if available)

list of variables and parameters for initialisation


\subsection{Input data}

\subsubsection{Weather data}
weather data from DWD (*.csv file)
climate data from GERICS (NetCDF file --> *.csv file) 

daily mean temperature
daily max temperature
daily min temperature
daily humidity


\subsubsection{Fructification data}
Fructification index of the European beech (Fagus sylvatica)
to estimate rodent density


\subsection{Submodels}

\subsubsection{Development}

\subsubsection{Desiccation}

\subsubsection{Freezing}

\subsubsection{Activity}

\subsubsection{Feeding}


\end{document}