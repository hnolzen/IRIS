\documentclass[a4paper, 11pt]{article}
\usepackage{geometry}
\geometry{top = 2cm, bottom = 2cm, left = 2cm, right = 2cm}
\usepackage[english]{babel}
\usepackage[utf8]{inputenc}
\usepackage[T1]{fontenc}
\usepackage{amsmath}
\usepackage[colorlinks = true, allcolors = blue]{hyperref}

\title{TRACE Modelling Notebook}
\author{}
\date{\today}

\begin{document}

\maketitle
\tableofcontents

\section{Master catalogue}
% A list of the locations of files most relevant to the project, with a description of the file and folder taxonomy.

\section{Work log}
% The main body of the notebook, composed of daily, dated entries. Each entry includes:


\subsection{Data Evaluation}

%(a) General information (common to all entries):
%(i) date of the entry,
%(ii) author of the entry,
%(iii) TRACE tag indicating the TRACE element the entry is linked to (Table 1),
%(iv) keyword indicating the specific modelling task within the TRACE element (Table 1),
%(v) title,
% (vi) overview of what has been done and what has been accomplished
%(vii) files linked to the entry (e.g., program code, script used to generate the experiment, spreadsheet containing
% parameter values, model input files, output files from experiments, summary files).

% (b) Specific details, which depend on the specific modelling task.

\end{document}