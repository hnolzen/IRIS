\documentclass[a4paper, 11pt, parskip = full]{scrartcl}
\usepackage[top = 2.5cm, bottom = 2cm, left = 2.5cm, right = 2.5cm]{geometry}
\usepackage[onehalfspacing]{setspace}
\usepackage{mathtools}
\usepackage{fontspec}
\usepackage[english]{babel}
\usepackage{hyperref}
\hypersetup{
	colorlinks = true,
	linkcolor  = black,
	citecolor  = blue,
	urlcolor   = blue
}

\usepackage{csquotes}
\usepackage[backend = biber, style = apa]{biblatex}
\addbibresource{../iris_literature.bib}


\title{IRIS - TRACE Documentation}
\author{}
\date{\today}

\begin{document}

\maketitle
\tableofcontents


\section{Problem formulation}
\textbf{This TRACE element provides supporting information on:} The decision-making context in which the model will
be used; the types of model clients or stakeholders addressed; a precise specification of the question(s) that should be
answered with the model, including a specification of necessary model outputs; and a statement of the domain of
applicability of the model, including the extent of acceptable extrapolations.

\textbf{SUMMARY:}
\begin{addmargin}[3em]{2em}
\textbf{

}
\end{addmargin}



\clearpage
\section{Model Description}
\textbf{This TRACE element provides supporting information on:} The model. Provide a detailed written model
description. For individual/agent-based and other simulation models, the ODD protocol is recommended as standard format. For complex
submodels it should include concise explanations of the underlying rationale. Model users should learn what the model
is, how it works, and what guided its design.

\textbf{SUMMARY:}
\begin{addmargin}[3em]{2em}
	\textbf{

	}
\end{addmargin}



\clearpage
\section{Data Evaluation}
\textbf{This TRACE element provides supporting information on:} The simplifying assumptions underlying a model's
design, both with regard to empirical knowledge and general, basic principles. This critical evaluation allows model users to
understand that model design was not ad hoc but based on carefully scrutinized considerations.

\textbf{SUMMARY:}
\begin{addmargin}[3em]{2em}
\textbf{

}
\end{addmargin}



\clearpage
\section{Conceptual model evaluation}
\textbf{This TRACE element provides supporting information on:} The simplifying assumptions underlying a model's
design, both  with regard to empirical knowledge and general, basic principles. This critical evaluation allows model
users to understand that model design was not ad hoc but based on carefully scrutinized considerations.

\textbf{SUMMARY:}
\begin{addmargin}[3em]{2em}
\textbf{

}
\end{addmargin}



\clearpage
\section{Implementation verification}
\textbf{This TRACE element provides supporting information on:} (1) whether the computer code implementing the model
has been thoroughly tested for programming errors, (2) whether the implemented model performs as indicated by the model
description, and (3) how the software has been designed and documented to provide necessary usability tools
(interfaces, automation of experiments, etc.) and to facilitate future installation, modification, and maintenance.

\textbf{SUMMARY:}
\begin{addmargin}[3em]{2em}
\textbf{

}
\end{addmargin}



\clearpage
\section{Model output verification}
\textbf{This TRACE element provides supporting information on:} (1) how well model output matches observations and
(2) how much calibration and effects of environmental drivers were involved in obtaining good fits of model output
and data.

\textbf{SUMMARY:}
\begin{addmargin}[3em]{2em}
\textbf{

}
\end{addmargin}



\clearpage
\section{Model analysis and application}
\textbf{This TRACE element provides supporting information on:} (1) how sensitive model output is to changes in model
parameters (sensitivity analysis), and (2) how well the emergence of model output has been understood.

\textbf{SUMMARY:}
\begin{addmargin}[3em]{2em}
\textbf{

}
\end{addmargin}



\clearpage
\section{Model output corroboration}
\textbf{This TRACE element provides supporting information on:} How model predictions compare to independent data and
patterns
that were not used, and preferably not even known, while the model was developed, parameterized, and verified. By
documenting model output corroboration, model users learn about evidence which, in addition to model output
verification, indicates that the model is structurally realistic so that its predictions can be trusted to some degree.

\textbf{SUMMARY:}
\begin{addmargin}[3em]{2em}
\textbf{

}
\end{addmargin}



\newpage
\printbibliography[heading = bibintoc, title = {Contents}]

\end{document}